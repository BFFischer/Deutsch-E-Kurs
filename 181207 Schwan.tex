\documentclass[12pt,a4paper]{scrartcl}        
\usepackage[T1]{fontenc} 
\usepackage[utf8]{inputenc} 
\usepackage[ngerman]{babel}
\usepackage{setspace}
\usepackage{lineno}
%
\begin{document} 
\author{Max Muster} 
\title{Einen satirischen Text analysieren} 
\maketitle 
\modulolinenumbers[5]
\begin{linenumbers}


\section{Der sterbende Schwan, Andrea Kümpfbeck 2003} 
Er liegt im Sterben. Wieder einmal. Seine letzten Tage will er allein verbringen. Er hat sich zurückgezogen - hinter einer Familienpackung Papiertaschentücher (extrazart für gereizte Schniefnasen), einer Zehn-Liter-Kanne Lindenblütentee (zum Schwitzen), drei Kilo Hustenbonbons (zum Im-Tee-Auflösen), zwei Wärmeflaschen (zum Wechseln), drei Flaschen Meerwasser Nasenspray (für die angegriffenen Schleimhäute), einem Körbchen Kiwis (für die Vitamine), einer Fleece-Decke (gegen Schüttelfrost) und vier Flaschen Hustensaft (mit Himbeergeschmack).

Er leidet. Wieder einmal. Ob die roten Flecken unter der Nase ein schlimmes Zeichen seien? Oder der geschwollende Lymphknoten auf der linken Halsseite? Zwischen all den Selbstdiagnosen, Gesundheitsratgeber - Sendungen und laut vorgelesenen Medizin - Lexika spurtet er immer wieder ins Bad. Weil es die Abwehrkräfte stärken soll, 18-mal täglich eine Handvoll lauwarmes Wasser durch die Nase in Richtung Stirnhöhle einzuatmen.

Die Telefonrechnung wird doppelt so hoch sein wie in gesunden Monaten. Das kennen wir schon. Ist auch verständlich, dass er sich achtmal täglich den Rat des Arztes einholen muss. Den interessieren die um 0,2 Grad gestiegene Temperatur und 28 verschnäuzte Taschentücher sicherlich. Der glühende Kopf, der drohende Rückschlag. Fieber hat er nämlich auch: 37,5 Grad Celsius. Der Arme! Das "'Arme"' ist wichtig: Wenn man im Abstand von zehn Minuten immer wieder bestätigt, wie schlecht es ihm geht und wie ungerecht diese Welt doch ist, trägt das wesentlich zum Gesundwerden bei. Außerdem: Es läuft nicht mehr - ohne ihn. Der Betrieb im Büro steht kurz vor dem Zusammenbruch, die Getränke sind längst ausgegangen, das Auto fährt schon lange nicht mehr. Drei Taschentuch-Großpackungen und 20 Liter Lindenblütentee später ist er immer noch nicht gestorben. Will er plötzlich auch nicht mehr. Mann ist erkältet. Wieder einmal.
\end{linenumbers}
\begin{spacing}{2}

\section{Einleitung}
% entspricht S. 205, Aufgabe 2a
% ab hier schreiben:

\section{Zusammenfassung}
% entspricht S. 205, Aufgabe 2b
% ab hier schreiben:


\section{Erklärungen}
% entspricht S. 205, Aufgabe 2c, bitte Zeilennummern angeben.
% ab hier schreiben:

\section{Die satirischen Mittel}
% entspricht S. 205, Aufgabe 2d
% ab hier schreiben:


\section{Persönliche Stellungnahme}
"'Ich finde die Darstellung des erkälteten Mannes gemein und viel zu einseitig. In Wirklichkeit jammern Männer überhaupt nicht so viel."'\\
% entspricht S. 205, Aufgabe 2e
% ab hier schreiben:

\end{spacing}
\end{document} 
