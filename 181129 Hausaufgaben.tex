%%%%%%%%%%%%%%%%%%%%%%%%%%%%%%%%%%%%%%%%%%%%%%%%%%%%%%%%%%%%%%%%%%%%%%%%%%%%%%%%%%%%%%%
%
% Name:        Vor- und Zunamen ersetzen (Max Mustermann)
% Texteingabe: Das Wort PLATZHALTER suchen, löschen und ab da schreiben.
% Abgabe:      181129
% Uhrzeit:     20.00 Uhr
% Dateiname:   Nachname Vorname 181129
%
% Textbeginn:  ab: PLATZHALTER, Zeilenangabe machen!
%
% E-Mail:      In die Betreffzeile kommt der Dateiname!
% Text:        Sehr geehrter Herr Fischer, 
%              in der Anlage finden Sie meine Hausaufgabe zum 181128
%
%              Mit freundlichen Grüßen
%              Max Mustermann
%
% P.S.         Alles, was anders bezeichnet ist oder geschrieben wird, schicke ich zurück.
%
%              B. Fischer  
%
%%%%%%%%%%%%%%%%%%%%%%%%%%%%%%%%%%%%%%%%%%%%%%%%%%%%%%%%%%%%%%%%%%%%%%%%%%%%%%%%%%%%%%%%
%hier bitte nichts verändern!
\documentclass[12pt,a4paper]{scrartcl}        
\usepackage[T1]{fontenc} 
\usepackage[utf8]{inputenc} 
\usepackage[ngerman]{babel}
\usepackage{setspace}
\usepackage{lineno}
%
\begin{document} 
\author{Max Mustermann} 
\title{Frauen sind eitel. Männer? Nie-!} 
\maketitle 
\modulolinenumbers[5]
%
\begin{linenumbers}
\section{Frauen sind eitel. Männer? Nie -!}
Das war in Hamburg, wo jede vernünftige Reiseroute aufzuhören hat, weil es die schönste Stadt Deutschlands ist - und es war vor dem dreiteiligen Spiegel. Der Spiegel stand in einem Hotel, das Hotel stand vor der Alster, der Mann stand vor dem Spiegel. Die Morgen-Uhr zeigte genau fünf Minuten vor einhalb zehn.

Der Mann war nur mit seinem Selbstbewusstsein bekleidet, und es war jenes Stadium eines Ferientages, wo man sich mit geradezu wollüstiger Langsamkeit anzieht, trödelt, Sachen im Zimmer umherschleppt, tausend überflüssige Dinge aus dem Koffer holt, sie wieder hineinpackt, Taschentücher zählt und sich überhaupt benimmt wie ein mittlerer Irrer: Es ist ein geschäftiges Nichtstun, und dazu sind ja die Ferien auch da. Der Mann stand vor dem Spiegel.

Männer sind nicht eitel. Frauen sind es. Alle Frauen sind eitel. Dieser Mand stand vor dem Spiegel, weil der dreiteilig war und weil der Mann zu Hause keinen solchen besaß. Nun sah er sich, Antinous mit dem Hängebauch, im dreiteiligen Spiegel und bemühte sich, sein Profil so kritisch anzusehen, wie seine egoistische Verliebtheit das zuließ ... eigentlich ... und nun richtete er sich ein wenig auf - eigentlich sah er doch sehr gut im Spiegel aus, wie -? Er strich sich mit gekreuzten Armen über die Haut, wie es die tun, die in ein Bad steigen wollen ... und bei dieser Betätigung sah sein linkes Auge ganz zufällig durch die dünne Gardine zum Fenster hinaus. Da stand etwas.

Es war eine enge Seitenstraße, und gegenüber, in gleicher Etagenhöhe, stand an einem Fenster eine Frau, eine ältere Frau, schien's, die hatte die drübige Gardine leicht zur Seite gerafft, den Arm hatte sie auf ein kleines Podest gelehnt, und sie stierte, starrte, glotzte, äugte gerade auf des Mannes gespiegelten Bauch. Allmächtiger.

Der erste Impuls hieß den Mann vom Spiegel zurücktreten, in die schützende Weite des Zimmers, gegen Sicht gedeckt. So ein Frauenzimmer. Aber es war doch eine Art Kompliment, das war unleugbar; denn wenn jede auch dergleichen vielleicht immer zu tun pflegte - es war eine Schmeichelei."'An die Schönheit."' Unleugbar war das so. Der Mann wagte sich drei Schritt vor.

Wahrhaftig: Da stand sie noch immer und äugte und starrte. Nut - man ist auf der Welt, um Gutes zu tun ... und wir können uns doch noch alle Tage sehen lassen - ein erneuter Blick in den Spiegel bestätigte das - heran an den Spiegel, heran ans Fenster!

Nein. Es war zu schéhnierlich .. der Mann hüpfte davon wie ein junges Mädchen, eilte ins Badezimmer und rasierte sich mit dem neuen Messer, das glitt sanft über die Haut wie ein nasses Handtuch, es war eine Freude. Abspülen ("`Scharf nachwaschen?"', fragte er sich selbst und bejahte es), scharf nachwaschen, pudern .. das dauerte gut und gern seine zehn Minuten. Zurück. Wollen doch spaßeshalber einmal sehen -.

Sie stand wahr und wahrhaftig noch immer da; in genau derselben Stellung wie vorhin stand sie da, die Gardine leicht zur Seite gerafft, den Arm aufgestützt, und sah regungslos herüber. Das war denn doch - also, dass wollen wir doch mal sehen. 

Der Mannn ging nun überhaupt nicht mehr vom Spiegel fort. Er machte sich dort zu schaffen, wie eine Bühnenzofe auf dem Theater: Er bürstete sich und legte einen Kamm von der rechten auf die linke Seite des Tischchens: er schnitt sich die Nägel und trocknete sich ausführlich hinter den Ohren, er sah sich prüfend von der Seite an, von vorn und auch sonst ... ein schiefer Blick über die Straße: Die Frau, die Dame, das Mädchen - sie stand noch immer da.

Der Mann, im Vollgefühl seiner maskulinen Siegerkraft, bewegte sich wie ein Gladiator im Zimmer, er tat so, als sei das Fenster nicht vorhanden, er ignorierte scheinbar ein Publikum, für das er alles tat: Er schlug ein Rad, und sein ganzer Körper machte fast hörbar: Kikeriki! Dann zog er sich, mit leisem Bedauern, an.

Nun war da ein manierlich bekleideter Herr - die Person stand doch immer noch da! -, er zog die Gardine zurück und öffnete mit leicht vertaulichem Lächeln das Fenster. Und sah hinüber.


Die Frau, vor der er eine halbe Stunde lang seine männliche Nacktheit produziert hatte, war - ein Holzgestell mit einem Mantel darüber, eine Zimmerpalme und ein dunkler Stuhl. So wie man im nächtlichen Wald aus Laubwerk und Ästen Gesichter komponiert, so hatte er eine Zuschauerin gesehen, wo nichts gewesen war als Holz, Stoff und eine Zimmerpalme.

Leicht begossen schloss der Herr Mann das Fenster. Frauen sind eitel. Männer -?

Männer sind es nie.\\
\end{linenumbers}
\textbf{Kurt Tucholsky (1928)}, aus: Klartext 10, S. 201f
%
\section{Textanalyse}
\begin{spacing}{2}
%Bitte den Text verbesserten Text aus dem Buch (S. 203f) fehlerfrei aufschreiben. Ab hier schreiben:
PLATZHALTER


\end{spacing}
\end{document} 
 
