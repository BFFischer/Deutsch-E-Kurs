\documentclass[12pt,a4paper]{scrartcl}        
\usepackage[T1]{fontenc} 
\usepackage[utf8]{inputenc} 
\usepackage[ngerman]{babel}
\usepackage{setspace}
\usepackage{lineno}
%
\begin{document} 
\author{Max Muster} 
\title{Einen satirischen Text analysieren} 
\maketitle 
\modulolinenumbers[5]
%
\begin{linenumbers}
\section{Grillen und Männer: Die Steinzeit lebt} 
"Darf ich mal?", fragen mich immer die Männer unter meinen Bekannten, die ich zum Grillen eingeladen habe. Jeder Mann weiß, wie man mit Feuer umgeht, und jeder weiß es besser als der andere. Während die Weibchen am Tisch interessiert zuschauen, kämpfen die Männchen um die Hoheit am Grill. Die Steinzeit lebt. Da Holzkohle bekanntlich zu den flammenhemmendsten und zündungswilligsten Stoffen der Welt gehört, ist das immer eine besondere Herausforderung für den männlichen Neandertaler bei der Fete. Ich gehöre zu der umstrittenen Fraktion, die das Problem politisch unkorrekt durch Zufuhr großer Mengen an Aktivierungsenergie (in Form mehrerer Liter Benzin) zu lösen sucht. Die meisten Männer suchen die Lösung hingegen in strategischem Vorgehen durch geschickte Platzierung von Anzündern und gezieltes Anhauchen der sich bildenden Glühfront. Dieses Verfahren, das von Umsicht und Klugheit zeugt, kann aber Stunden dauern, und ich habe Hunger. Deshalb mache ich stets von meinem Recht als Gastgeber und Hausherr Gebrauch und initiiere den Grillabend mit einem imposanten Feuerball. Da bin ich dann zwar für den Augenblick das erfolgreichste Männchen am Platz, aber die rußgeschwärzten Weibchen danken es mir wieder mal nicht.\\
\end{linenumbers}
Harald Mohr, 2003 aus: Lesebuch S. 206
\begin{spacing}{2}
\section{Einleitung}
% TATTE-Satz
% ab hier schreiben:

\section{Zusammenfassung}
% ab hier schreiben:

\section{Die satirischen Mittel}
% ab hier schreiben:


\section{Persönliche Stellungnahme}
% ab hier schreiben:

\end{spacing}
\end{document} 

