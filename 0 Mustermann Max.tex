\documentclass[12pt,a4paper]{scrartcl}        
\usepackage[T1]{fontenc} 
\usepackage[utf8]{inputenc} 
\usepackage[ngerman]{babel}
% Die 0 habe ich absichtlich gewählt.
% Das gilt auch für den Dateinamen: 0MustermannMax
\title{0 Inhaltsverzeichnis}
\author{Max Mustermann}
\pagenumbering{gobble}
\newcommand{\linia}{\rule{\linewidth}{0.5pt}}
\makeatletter
\renewcommand{\maketitle}{\begin{center}
\huge \@title\end{center}
\linia\\
{\large\@author\hfill\@date\\}}
\begin{document}
\maketitle
%%%%%%%%%%%%%%%%%%%%%%%%%%%%%%%%%%%%%%%%%%%%%%%%%%%%%%%%%%%%%%%%%%%%%%%%%%%%%%%%%%
% zur Vorlage bei Notenkonferenzen und beim nächsten Elternsprechtag.
% Bitte aktualisiert diese Liste ständig.  
% Ich behalte mir vor, zwischendurch individuelle Abfragen zu machen. B.F.
Meine Hausaufgaben % Nicht verändern!
%%%%%%%%%%%%%%%%%%%%%%%%%%%%%%%%%%%%%%%%%%%%%%%%%%%%%%%%%%%%%%%%%%%%%%%%%%%%%%%%%%
% Als Kapitelüberschrift soll hier das Datum, gefolgt einer Überschrift erscheinen.
\section{18???? Zirkuskarikatur}
\begin{itemize}
    % Quellenangabe
    \item Lesebuch, S. 190 
    % Den Inhalt kurz angeben:
    % Das hat mir ganz gut gefallen, weil ...
    % Hier habe ich den Sinn zunächst nicht verstanden, denn ...
    \item Ein Artistenehepaar erörtert während seiner Zirkusvorstellung das Rollenverhalten von Mann und Frau.
    % Eine kurze Bewertung oder einen Kommentar formulieren, z.B.
    \item Die Karikatur hat mir ganz gut gefallen, weil PLATZHALTER
\end{itemize}
%----------------------------------------------------------------------------------
\section{18???? Frauen kommen langsam, aber gewaltig}
\begin{itemize}
    \item PLATZHALTER
    \item PLATZHALTER
    \item PLATZHALTER
\end{itemize}
%----------------------------------------------------------------------------------
\section{xxxxxx Platzhalter}
\begin{itemize}
    \item PLATZHALTER
    \item PLATZHALTER
    \item PLATZHALTER
\end{itemize}
%----------------------------------------------------------------------------------
\section{?????? Platzhalter}
\begin{itemize}
    \item PLATZHALTER
    \item PLATZHALTER
    \item PLATZHALTER
\end{itemize}
%----------------------------------------------------------------------------------
\end{document}
